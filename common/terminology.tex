\newcommand{\Ms}{Master's thesis\xspace}

\newcommand{\premiseselection}{premise selection\xspace}
\newcommand{\Premiseselection}{Premise selection\xspace}

\newcommand{\cpp}{C++\xspace}
\newcommand{\ocaml}{OCaml\xspace}
\newcommand{\python}{Python\xspace}
\newcommand{\matlab}{MATLAB\xspace}
\newcommand{\xml}{XML\xspace}
\newcommand{\roerei}{\texttt{roerei}\xspace}

\newcommand{\coq}{Coq\xspace}
\newcommand{\gallina}{Gallina\xspace}
\newcommand{\coqide}{CoqIDE\xspace}

\newcommand{\machinelearning}{Machine Learning\xspace}
\newcommand{\crossvalidation}{cross validation\xspace}
\newcommand{\dagraph}{directed acyclic graph\xspace}
\newcommand{\prop}{\texttt{Prop}\xspace}
\newcommand{\cprop}{\texttt{CProp}\xspace}
\newcommand{\kindtype}{\texttt{Type}\xspace}

\newcommand{\compcert}{CompCert\xspace}
\newcommand{\corn}{CoRN\xspace}
\newcommand{\formalin}{Formalin\xspace}
\newcommand{\mathcomp}{Mathematical Components\xspace}

\newcommand{\knn}{$k$-Nearest Neighbor\xspace}
\newcommand{\nb}{Naive Bayes\xspace}
\newcommand{\ensemble}{Ensemble Learning\xspace}

\newglossaryentry{termclass}{name={term},
description={a noun or compound word of \pcic, used by \gallina}}
\newglossaryentry{typeclass}{name={type},
description={the semantic subclass of types inside the syntactic class term}}
\newglossaryentry{sort}{name={sort},
description={the type of a type when manipulated as term}}

\newcommand{\sortprop}{\texttt{Prop}\xspace}
\newglossaryentry{sortprop}{name={\sortprop},
description={the type of a type when manipulated as term}}
\newcommand{\sortset}{\texttt{Set}\xspace}
\newglossaryentry{sortset}{name={\sortset},
description={the type of a type when manipulated as term}}

\newcommand{\cic}{Cic\xspace}
\newglossaryentry{cic}{name={\cic},
description={Calculus of (Co)Inductive Constructions}}

\newcommand{\acic}{aCic\xspace}
\newglossaryentry{acic}{name={\acic},
description={Calculus of (Co)Inductive Constructions with Explicit Named Substitutions}}

\newcommand{\pcic}{pCic\xspace}
\newglossaryentry{pcic}{name={\pcic},
description={Predicative Calculus of (Co)Inductive Constructions}}

\newcommand{\coqobj}[1][]{\coq object ${#1}$\xspace}
\newglossaryentry{coqobj}{name={\coqobj[s]},
description={an object defined in \coq. Might be an inductive type, an axiom or a proof}}

\newcommand{\name}[1][]{n_{#1}}
\newglossaryentry{name}{name={\ensuremath{\name[s]}},
description={name of the \coqobj[s] as written in the Coq source file}}

\newcommand{\body}[1][]{t_{#1}}
\newglossaryentry{body}{name={\ensuremath{\body[s]}},
description={set of names of all objects in the proof body of $s$}}

\newcommand{\emptyterm}{\top}

\newcommand{\type}[1][]{A_{#1}}
\newglossaryentry{type}{name={\ensuremath{\type[s]}},
description={set of names of all objects in the proof type of $s$}}

\newcommand{\defs}[1][]{\texttt{Defs}_{}}
\newglossaryentry{defs}{name={\ensuremath{\defs}},
description={set of all names occuring in proof types}}
\newcommand{\thms}[1][]{\texttt{Thms}_{}}
\newglossaryentry{thms}{name={\ensuremath{\thms}},
description={set of all names occuring in proof bodies, excluding all $\defs$}}

\newcommand{\deps}[1][]{D_{#1}}
\newcommand{\depstrans}[1][]{\deps[{#1}]^*}
\newcommand{\parents}[1][]{P_{#1}}
\newcommand{\parentstrans}[1][]{\parents[{#1}]^*}
\newcommand{\features}[1][]{F_{#1}}
\newcommand{\featurekeys}{Z}
\newcommand{\objs}[1][]{S_{#1}}
\newcommand{\trainset}{\objs[{\text{train}}]}
\newcommand{\testset}{\objs[{\text{test}}]}
\newcommand{\counttype}[2]{\#_{#2}(\text{type of}~{#1})}
\newcommand{\countbody}[2]{\#_{#2}(\text{term of}~{#1})}

\newcommand{\objdef}{:=}

\newcommand{\citationeeded}{[citation needed]}
