\newcommand{\Ms}{Master's thesis\xspace}
\newcommand{\citationeeded}{[citation needed]}

% Generic
\newcommand{\machinelearning}{Machine Learning\xspace}
\newcommand{\crossvalidation}{cross validation\xspace}
\newcommand{\dagraph}{directed acyclic graph\xspace}
\newcommand{\premiseselection}{premise selection\xspace}
\newcommand{\Premiseselection}{Premise selection\xspace}

% Math
\newcommand{\nat}{\mathbb{N}}
\newcommand{\rat}{\mathbb{R}}

\newcommand{\powerset}[1]{\ensuremath{\mathcal{P}({#1})}}

\newcommand{\infimum}{\bot}
\newglossaryentry{infimum}{name={\ensuremath{\infimum}},
description={greatest element of a poset that is less than or equal to all elements of that poset}}

\newcommand{\nth}[2]{\text{nth}({#1}, {#2})}
\newglossaryentry{nth}{name={\ensuremath{\nth{n}{X}}},
description={the $n^{\text{th}}$ bottom element in a totally ordered set}}

\newcommand{\downset}[2]{~{#1}\downarrow{#2}~}
\newglossaryentry{downset}{name={\ensuremath{\downset{X}{x}}},
description={subset of a poset $X$ where each element is less than or equal to a given element $x$}}

% Lists
\newcommand{\listtype}[1]{\ensuremath{\mathcal{L}({#1})}}
\newcommand{\nil}{\ensuremath{\text{nil}}}
\newcommand{\cons}[2]{\ensuremath{\text{cons}({#1},~{#2})}}

\newcommand{\mapsymbol}{\ensuremath{\text{map}}}
\newcommand{\map}[2]{\ensuremath{\mapsymbol( {#1},~{#2})}}

\newcommand{\concatsymbol}{\ensuremath{\uplus}}
\newcommand{\concat}[2]{{#1} ~\concatsymbol~ {#2}}

\newcommand{\flattenlistsym}{\ensuremath{\text{flatten}}}
\newcommand{\flattenlist}[1]{\ensuremath{\flattenlistsym({#1})}}

\newcommand{\foldrsym}{\ensuremath{\text{foldr}}}
\newcommand{\foldr}[3]{\ensuremath{\foldrsym({#1},~ {#2},~ {#3})}}

\newcommand{\singleton}[1]{\ensuremath{[ {#1} ]}}

% Languages
\newcommand{\cpp}{C++\xspace}
\newcommand{\ocaml}{OCaml\xspace}
\newcommand{\python}{Python\xspace}
\newcommand{\matlab}{MATLAB\xspace}
\newcommand{\xml}{XML\xspace}

% Projects
\newcommand{\preloader}{\texttt{preloader}\xspace}
\newcommand{\roerei}{\texttt{roerei}\xspace}
\newcommand{\coq}{Coq\xspace}
\newcommand{\coqide}{CoqIDE\xspace}

% Corpora
\newcommand{\compcert}{CompCert\xspace}
\newcommand{\corn}{CoRN\xspace}
\newcommand{\formalin}{Formalin\xspace}
\newcommand{\mathclasses}{Mathematical Classes\xspace}
\newcommand{\mathcomp}{Mathematical Components\xspace}

% Machine learning methods
\newcommand{\knn}{$k$-Nearest Neighbor\xspace}
\newcommand{\knnadaptive}{Adaptive Nearest Neighbor\xspace}
\newcommand{\nb}{Naive Bayes\xspace}
\newcommand{\ensemble}{Ensemble Learning\xspace}
\newcommand{\omniscient}{Omniscient\xspace}
\newcommand{\mepo}{the MePo relevance filter\xspace}
\newcommand{\adarank}{Adarank\xspace}

% Coq terminology
\newcommand{\cprop}{\texttt{CProp}\xspace}

\newglossaryentry{termclass}{name={term},
description={a noun or compound word of \pcic, used by \gallina}}
\newglossaryentry{typeclass}{name={type},
description={the semantic subclass of types inside the syntactic class term}}
\newglossaryentry{sentence}{name={sentence},
description={a named term along with the corresponding type}}
\newglossaryentry{sort}{name={sort},
description={the type of a type when manipulated as term}}

\newcommand{\sorts}{\texttt{Sorts}\xspace}
\newglossaryentry{sorts}{name={\ensuremath{\sorts}}, sort=sorts,
description={the set of sorts}}
\newcommand{\sortprop}{\texttt{Prop}\xspace}
\newglossaryentry{sortprop}{name={\sortprop}, sort=prop,
description={the type of logical propositions}}
\newcommand{\sortset}{\texttt{Set}\xspace}
\newglossaryentry{sortset}{name={\sortset}, sort=set,
description={the type of small sets}}
\newcommand{\sorttype}[1][]{\texttt{Type}\ifthenelse{\equal{#1}{}}{}{({#1})}\xspace}
\newglossaryentry{sorttype}{name={\sorttype},
description={the type of types}}

\newcommand{\cic}{Cic\xspace}
\newglossaryentry{cic}{name={\cic},
description={Calculus of (Co)Inductive Constructions}}

\newcommand{\acic}{aCic\xspace}
\newglossaryentry{acic}{name={\acic},
description={Calculus of (Co)Inductive Constructions with Explicit Named Substitutions}}

\newcommand{\pcic}{pCic\xspace}
\newglossaryentry{pcic}{name={\pcic},
description={Predicative Calculus of (Co)Inductive Constructions}}

\newcommand{\gallina}{Gallina\xspace}
\newglossaryentry{gallina}{name={\gallina},
description={the specification language for \coq}}

% Paper-bound definitions
\newcommand{\coqobj}[1][]{\coq object\ifthenelse{\equal{#1}{}}{}{~${#1}$}\xspace}
\newcommand{\coqobjs}{\coq objects\xspace}
\newglossaryentry{coqobj}{name={\coqobj[s]},
description={an object defined in \coq. Might be an inductive type, an axiom or a proof}}

\newglossaryentry{theorem}{name=theorem,
description={\coqobjs that have been proved based on previously established \coqobjs}}

\newglossaryentry{definition}{name=definition,
description={\coqobjs that are not theorems. Either simple constants or purely transformative operations on such constants}}

\newcommand{\name}[1][]{n_{#1}}
\newglossaryentry{name}{name={\ensuremath{\name[s]}},
description={canonical form of the name of the \coqobj[s]}}

\newcommand{\names}{\mathcal{N}}
\newglossaryentry{names}{name={\ensuremath{\names}},
description={the set of all names}}

\newcommand{\term}[1][]{t_{#1}}
\newglossaryentry{term}{name={\ensuremath{\term[s]}},
description={term of the \coqobj[s]}}

\newcommand{\terms}{\mathcal{T}}
\newglossaryentry{terms}{name={\ensuremath{\terms}},
description={the set of all terms}}

\newcommand{\emptyterm}{\top}
\newglossaryentry{emptyterm}{name={\ensuremath{\emptyterm}}, sort=term-empty,
description={the empty term}}

\newcommand{\type}[1][]{A_{#1}}
\newglossaryentry{type}{name={\ensuremath{\type[s]}},
description={type of the \coqobj[s]}}

\newcommand{\types}{\mathcal{A}}
\newglossaryentry{types}{name={\ensuremath{\types}},
description={the set of all types}}

\newcommand{\flattensym}{\nabla}
\newcommand{\flatten}[1]{\flattensym({#1})}
\newglossaryentry{flatten}{name={\ensuremath{\flattensym}}, sort=flatten,
description={extracts names from term or type, and return the set of names}}

\newcommand{\countsym}{\#}

\newcommand{\countoccursym}{\flattensym_\countsym}
\newcommand{\countoccur}[1]{\countoccursym({#1})}
\newglossaryentry{countoccur}{name={\ensuremath{\countoccursym}}, sort=flatten-countoccur,
description={extracts names from term or type along with their number of occurances}}

\newcommand{\depthsym}{D}

\newcommand{\depthoccursym}{\flattensym_\depthsym}
\newcommand{\depthoccur}[1]{\depthoccursym({#1})}
\newglossaryentry{depthoccur}{name={\ensuremath{\depthoccursym}}, sort=flatten-depthoccur,
description={extracts names from term or type along with their depth}}

\newcommand{\termset}[1]{\flatten{\term[#1]}}
\newglossaryentry{termset}{name={\ensuremath{\termset{s}}}, sort=flatten-termset,
description={set of names of all objects in the term of \coqobj[s]}}

\newcommand{\typeset}[1]{\flatten{\type[#1]}}
\newglossaryentry{typeset}{name={\ensuremath{\typeset{s}}}, sort=flatten-typeset,
description={set of names of all objects in the type of \coqobj[s]}}

\newcommand{\defs}[1][]{\texttt{Defs}_{#1}}
\newglossaryentry{defs}{name={\ensuremath{\defs}},
description={set of names of all objects in the types of all objects; used as the base of the set of features}}
\newcommand{\thms}[1][]{\texttt{Thms}_{#1}}
\newglossaryentry{thms}{name={\ensuremath{\thms}},
description={set of names of all objects in the terms of all objects, excluding all $\defs$; used as set of dependencies}}

% Metrics
\newcommand{\oocover}{100Cover\xspace}
\newcommand{\oocoverf}[2]{\ensuremath{\text{100Cover}({#1}, {#2})}}
\newcommand{\ooprecision}{100Precision\xspace}
\newcommand{\ooprecisionf}[2]{\ensuremath{\text{100Precision}({#1}, {#2})}}
\newcommand{\recall}{Recall\xspace}
\newcommand{\recallf}[2]{\ensuremath{\text{Recall}({#1}, {#2})}}
\newcommand{\rank}{Rank\xspace}
\newcommand{\rankf}[2]{\ensuremath{\text{Rank}({#1}, {#2})}}
\newcommand{\auc}{AUC\xspace}
\newcommand{\aucf}[2]{\ensuremath{\text{AUC}({#1}, {#2})}}
\newcommand{\volume}{Volume\xspace}

% Features
\newcommand{\featurekeys}{Z}
\newglossaryentry{featurekeys}{name={\ensuremath{\featurekeys}},
description={set of feature keys, or the set of the aspects with which an object can be described; based on the set of definitions $\defs$, but possibly extended with other aspects}}

\newcommand{\features}[2]{F^{#1}_{#2}}
\newglossaryentry{features}{name={\ensuremath{\features{}{}}},
description={function which computes the numeric values describing a \coqobj. These values are considered to be the features of the \coq object}}

\newcommand{\depsym}{D}

\newcommand{\deps}[1]{\depsym_{#1}}
\newglossaryentry{deps}{name={\ensuremath{\deps{s}}},
description={set of names of objects which have been used to define or prove a \coqobj $s$}}

\newcommand{\depset}{\depsym}
\newglossaryentry{depset}{name={\ensuremath{\depset}},
description={set of all possible dependencies}}

\newcommand{\depstrans}[1][]{\deps{#1}^*}

\newcommand{\parentsym}{P}

\newcommand{\parents}[1][]{\parentsym_{#1}}
\newglossaryentry{parentsym}{name={\ensuremath{\parents{s}}},
description={set of names of objects which use \coqobj $s$ to be defined to or proven}}

\newcommand{\parentstrans}[1][]{\parents[{#1}]^*}

% Predictors
\newcommand{\predictors}{\mathcal{P}}
\newglossaryentry{predictors}{name={\ensuremath{\predictors}},
description={set of predictor functions}}

\newcommand{\rankings}{\mathcal{R}}
\newglossaryentry{rankings}{name={\ensuremath{\rankings}},
description={set of rankings}}

\newcommand{\dist}{\text{dist}}
\newglossaryentry{dist}{name={\ensuremath{\dist}},
description={the euclidean distance between two types $a, b \in \types$ computed using the features $\features{}{a}$ and $\features{}{b}$ of those types}}

\newcommand{\objsym}{\ensuremath{S}}
\newcommand{\objs}[1][]{\ensuremath{\objsym_{#1}}}
\newcommand{\trainset}{\objs[{\text{train}}]}
\newcommand{\testset}{\objs[{\text{test}}]}

\newcommand{\objdef}{:=}

\newcommand{\findindex}[2]{\ensuremath{i}({#1}, {#2})}
\newcommand{\firstsym}{\ensuremath{\text{first}}}
\newcommand{\topn}[2]{\ensuremath{\mathtt{top}_{#1}({#2})}}
\newcommand{\required}[1][]{\ensuremath{{\mathtt{req}}_{#1}}}
\newcommand{\suggestions}[1]{\ensuremath{\mathtt{sugg}_{#1}}}

\newcommand{\ltr}{\emph{Learning to rank}\xspace}

% Adarank

\newcommand{\query}{\ensuremath{q}}
\newcommand{\doc}{\ensuremath{d}}
\newcommand{\docs}[1]{\ensuremath{\mathbf{d}_{#1}}}
\newcommand{\queries}{\ensuremath{Q}}
\newcommand{\rankdoc}[1]{\ensuremath{y}_{#1}}
\newcommand{\irfeatures}{\ensuremath{\mathcal{X}}}

\newcommand{\idf}{\ensuremath{\mathbf{idf}}}
