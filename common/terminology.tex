\newcommand{\Ms}{Master's thesis\xspace}
\newcommand{\citationeeded}{[citation needed]}

% Generic
\newcommand{\machinelearning}{Machine Learning\xspace}
\newcommand{\crossvalidation}{cross validation\xspace}
\newcommand{\dagraph}{directed acyclic graph\xspace}
\newcommand{\premiseselection}{premise selection\xspace}
\newcommand{\Premiseselection}{Premise selection\xspace}

% Languages
\newcommand{\cpp}{C++\xspace}
\newcommand{\ocaml}{OCaml\xspace}
\newcommand{\python}{Python\xspace}
\newcommand{\matlab}{MATLAB\xspace}
\newcommand{\xml}{XML\xspace}

% Projects
\newcommand{\preloader}{\texttt{preloader}\xspace}
\newcommand{\roerei}{\texttt{roerei}\xspace}
\newcommand{\coq}{Coq\xspace}
\newcommand{\coqide}{CoqIDE\xspace}

% Corpora
\newcommand{\compcert}{CompCert\xspace}
\newcommand{\corn}{CoRN\xspace}
\newcommand{\formalin}{Formalin\xspace}
\newcommand{\mathcomp}{Mathematical Components\xspace}

% Machine learning methods
\newcommand{\knn}{$k$-Nearest Neighbor\xspace}
\newcommand{\nb}{Naive Bayes\xspace}
\newcommand{\ensemble}{Ensemble Learning\xspace}

% Coq terminology
\newcommand{\cprop}{\texttt{CProp}\xspace}

\newglossaryentry{termclass}{name={term},
description={a noun or compound word of \pcic, used by \gallina}}
\newglossaryentry{typeclass}{name={type},
description={the semantic subclass of types inside the syntactic class term}}
\newglossaryentry{sentence}{name={sentence},
description={a named term along with the corresponding type}}
\newglossaryentry{sort}{name={sort},
description={the type of a type when manipulated as term}}

\newcommand{\sorts}{\texttt{Sorts}\xspace}
\newglossaryentry{sorts}{name={\ensuremath{\sorts}}, sort=sorts,
description={the set of sorts}}
\newcommand{\sortprop}{\texttt{Prop}\xspace}
\newglossaryentry{sortprop}{name={\sortprop}, sort=prop,
description={the type of logical propositions}}
\newcommand{\sortset}{\texttt{Set}\xspace}
\newglossaryentry{sortset}{name={\sortset}, sort=set,
description={the type of small sets}}
\newcommand{\sorttype}[1][]{\texttt{Type}\ifthenelse{\equal{#1}{}}{}{({#1})}\xspace}
\newglossaryentry{sorttype}{name={\sorttype},
description={the type of types}}

\newcommand{\cic}{Cic\xspace}
\newglossaryentry{cic}{name={\cic},
description={Calculus of (Co)Inductive Constructions}}

\newcommand{\acic}{aCic\xspace}
\newglossaryentry{acic}{name={\acic},
description={Calculus of (Co)Inductive Constructions with Explicit Named Substitutions}}

\newcommand{\pcic}{pCic\xspace}
\newglossaryentry{pcic}{name={\pcic},
description={Predicative Calculus of (Co)Inductive Constructions}}

\newcommand{\gallina}{Gallina\xspace}
\newglossaryentry{gallina}{name={\gallina},
description={the specification language for \coq}}

% Paper-bound definitions
\newcommand{\coqobj}[1][]{\coq object\ifthenelse{\equal{#1}{}}{}{~${#1}$}\xspace}
\newcommand{\coqobjs}{\coq objects\xspace}
\newglossaryentry{coqobj}{name={\coqobj[s]},
description={an object defined in \coq. Might be an inductive type, an axiom or a proof}}

\newglossaryentry{theorem}{name=theorem,
description={\coqobjs that have been proved based on previously established \coqobjs}}

\newglossaryentry{definition}{name=definition,
description={\coqobjs that are not theorems. Either simple constants or purely transformative operations on such constants}}

\newcommand{\name}[1][]{n_{#1}}
\newglossaryentry{name}{name={\ensuremath{\name[s]}},
description={canonical form of the name of the \coqobj[s]}}

\newcommand{\names}{\mathcal{N}}
\newglossaryentry{names}{name={\ensuremath{\names}},
description={the set of all names}}

\newcommand{\term}[1][]{t_{#1}}
\newglossaryentry{term}{name={\ensuremath{\term[s]}},
description={term of the \coqobj[s]}}

\newcommand{\terms}{\mathcal{T}}
\newglossaryentry{terms}{name={\ensuremath{\terms}},
description={the set of all terms}}

\newcommand{\emptyterm}{\top}
\newglossaryentry{emptyterm}{name={\ensuremath{\emptyterm}}, sort=term-empty,
description={the empty term}}

\newcommand{\type}[1][]{A_{#1}}
\newglossaryentry{type}{name={\ensuremath{\type[s]}},
description={type of the \coqobj[s]}}

\newcommand{\types}{\mathcal{A}}
\newglossaryentry{types}{name={\ensuremath{\types}},
description={the set of all types}}

\newcommand{\flattensym}{\nabla}
\newcommand{\flatten}[1]{\flattensym({#1})}
\newglossaryentry{flatten}{name={\ensuremath{\flattensym}}, sort=flatten,
description={extracts names from term or type, and return the set of names}}

\newcommand{\countoccursym}{\flattensym_\#}
\newcommand{\countoccur}[1]{\countoccursym({#1})}
\newglossaryentry{countoccur}{name={\ensuremath{\countoccursym}}, sort=flatten-countoccur,
description={extracts names from term along with their number of occurances in the term}}

\newcommand{\depthsym}{\flattensym_D}
\newcommand{\depth}[1]{\depthsym({#1})}
\newglossaryentry{depth}{name={\ensuremath{\depthsym}}, sort=flatten-depth,
description={extracts names from term along with their depth in the term}}

\newcommand{\termset}[1]{\flatten{\term[#1]}}
\newglossaryentry{termset}{name={\ensuremath{\termset{s}}}, sort=flatten-termset,
description={set of names of all objects in the term of \coqobj[s]}}

\newcommand{\typeset}[1]{\flatten{\type[#1]}}
\newglossaryentry{typeset}{name={\ensuremath{\typeset{s}}}, sort=flatten-typeset,
description={set of names of all objects in the type of \coqobj[s]}}

\newcommand{\defs}[1][]{\texttt{Defs}_{}}
\newglossaryentry{defs}{name={\ensuremath{\defs}},
description={set of names of all objects in the types of all objects; used as set of features}}
\newcommand{\thms}[1][]{\texttt{Thms}_{}}
\newglossaryentry{thms}{name={\ensuremath{\thms}},
description={set of names of all objects in the terms of all objects, excluding all $\defs$; used as set of dependencies}}

\newcommand{\oocover}{100Cover\xspace}
\newcommand{\ooprecision}{100Precision\xspace}
\newcommand{\recall}{Recall\xspace}
\newcommand{\rank}{Rank\xspace}
\newcommand{\auc}{AUC\xspace}
\newcommand{\volume}{Volume\xspace}

\newcommand{\deps}[1][]{D_{#1}}
\newcommand{\depstrans}[1][]{\deps[{#1}]^*}
\newcommand{\parents}[1][]{P_{#1}}
\newcommand{\parentstrans}[1][]{\parents[{#1}]^*}
\newcommand{\features}[1][]{F_{#1}}
\newcommand{\featurekeys}{Z}
\newcommand{\objs}[1][]{S_{#1}}
\newcommand{\trainset}{\objs[{\text{train}}]}
\newcommand{\testset}{\objs[{\text{test}}]}
\newcommand{\counttype}[2]{\#_{#2}(\text{type of}~{#1})}
\newcommand{\countbody}[2]{\#_{#2}(\text{term of}~{#1})}

\newcommand{\objdef}{:=}
