% No ATP tool for Coq exists currently.
% Many ATP tools out there.
% Machine Learning proven viable as base for ATP.

\subsection{\coq}

In order to perform \premiseselection for \coq we first need to examine what definitions in \coq look like, and what we exactly want to predict.
A proof in \coq is comprised of one or more files written in the \gallina language \cite{huet1992gallina} .
For a complete definition of the \gallina language please refer to the \coq reference manual\footnote{\coq Reference Manual, Chapter 1: The Gallina specification language\\ \url{https://coq.inria.fr/refman/Reference-Manual003.html}}.

Each file is, amongst other things, comprised of sentences.

Internally \coq represents all objects as instances within \cic, or more recently \pcic.

\begin{definition}[\cic]\defgls{cic}
	Calculus of (Co)Inductive Constructions,
	or the Calculus of Constructions \cite{coquand1988calculus} with inductive types and coinduction \cite{huet1987induction} \cite{coquand1990inductively}.
	Used for \coq v7.
	\todo{give full definition}
\end{definition}

\begin{definition}[\pcic]\defgls{pcic}
	Predicative Calculus of (Co)Inductive Constructions \cite{bertot2013interactive}.
	Used for \coq v8.
\end{definition}

\begin{definition}[{\coqobj[s]}]\defgls{coqobj}
	\coqobj[s] of the form $\hname :\equiv \body : \type$.
\end{definition}


