For the \coq interactive theorem prover automation is limited to the following tactics:
\begin{description}
\item[assumption]
  \footnote{
    \coq Reference Manual, Chapter 8:
    Tactics\\
    \url{https://coq.inria.fr/refman/Reference-Manual010.html\#hevea_tactic5}
  }
  Searches in the local context for an hypothesis which fits the current goal.
\item[auto]
  \footnote{
    \coq Reference Manual, Chapter 8:
    Tactics\\
    \url{https://coq.inria.fr/refman/Reference-Manual010.html\#hevea_tactic148}
  }
  Implements a Prolog-like resolution procedure to solve the current goal.
  It first applies the \texttt{assumption} tactic, and reduces the goal using \texttt{intros}.
  \texttt{auto} then runs a search for applicable tactics on the goal and subgoals, starting with the lowest cost tactic.
  This tactic can be tweaked on with various parameters and hints, such as the search depth.
\item[trivial]
  \footnote{
    \coq Reference Manual, Chapter 8:
    Tactics\\
    \url{https://coq.inria.fr/refman/Reference-Manual010.html\#hevea_tactic150}
  }
  A restriction of \texttt{auto} that is not recursive and only applies tactics with no cost.
\item[tauto]
  \footnote{
    \coq Reference Manual, Chapter 8:
    Tactics\\
    \url{https://coq.inria.fr/refman/Reference-Manual010.html\#hevea_tactic154}
  }
  A decision procedure for intuitionistic propositional calculus.
  Restricted to unfolding negations and logical equivalence.
  \texttt{tauto} is based on LJT* calculus by Dyckhoff et al \cite{dyckhoff1992contraction}.
\item[intuition \emph{tactic}]
  \footnote{
    \coq Reference Manual, Chapter 8:
    Tactics\\
    \url{https://coq.inria.fr/refman/Reference-Manual010.html\#hevea_tactic156}
  }
  Generates simplified subgoals and applies the provided tactic.
  \text{intuition} succeeds if the provided tactic succeeds on all (simpler) subgoals, else it fails.
  It is based on the decision procedure used in \texttt{tauto}.
  The expression \texttt{intuition fail} is equivalent to \texttt{tauto}.
\item[omega]
  an automatic decision procedure for Presburger arithmetic \cite{stansifer1984presburger}.%
  \footnote{
    \coq Reference Manual, Chapter 21:
    Omega: a solver of quantifier-free problems in Presburger Arithmetic\\
    \url{https://coq.inria.fr/refman/Reference-Manual024.html\#hevea_tactic220}
  }
\end{description}

In this masters thesis we explore the feasibility of implementing a statistical automated theorem prover for \coq.
\todo{statistical vs symbolical ATP; explain the difference}
\todo{explain `hammer' approach and what use it is for premiseselection}
Such an automated theorem prover consists of:
\begin{description}
\item[Premise selection] Filter the existing knowledgebase of proven theorems by usefullness for the current yet unproven conjecture.
It yields a set of theorems which are likely to help in proving this conjecture.
Subsequent ATP components thus will only need to consider useful theorems. \todo{explain}
\item[]
\end{description}

Because this endeavour is quite sizeable, we limit the research to implementing a \premiseselection tool.
We compare various machine learning algorithms and features for a variety of existing \coq corpora.
