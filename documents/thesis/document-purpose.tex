% Implement Premise selection for Coq.
% Determine viability of approach on different Coq corpora.

% Only go as far as Premise selection, as this is only a masters' thesis.
% Could dedicate a few PhD positions to this.
For the \coq interactive theorem prover automation is limited to the tactics \texttt{auto}%
\footnote{\coq Reference Manual, Chapter 8: Tactics\\\url{https://coq.inria.fr/refman/Reference-Manual010.html\#hevea_tactic148}},
which implements a Prolog-like resolution procedure, and \texttt{omega}%
\footnote{\coq Reference Manual, Chapter 21: Omega: a solver of quantifier-free problems in Presburger Arithmetic\\\url{https://coq.inria.fr/refman/Reference-Manual024.html\#hevea_tactic220}},
which is an automatic decision procedure for Presburger arithmetic.
In this masters thesis we explore the feasibility of implementing a statistical automated theorem prover for \coq.
Such an automated theorem prover consists of:
\begin{description}
\item[Premise selection] Filter the existing knowledgebase of proven theorems by usefullness for the current yet unproven conjecture.
It yields a set of definitions and theorems which are likely to help in proving this conjecture.
Subsequent ATP components thus will only need to consider useful theorems.
\item[]
\end{description}

Because this endeavour is quite sizeable, we limit the research to implementing a \premiseselection tool.
We compare various machine learning algorithms and features for a variety of existing \coq corpora.
