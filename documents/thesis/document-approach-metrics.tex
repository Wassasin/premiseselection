Given a predictor $P$ and a test object $s \in \testset$,
the performance of the predictor for this conjecture can be computed.
The general performance of the predictor for a given $\testset$
can be computed by averaging the values of a metric for all elements in $\testset$.

\begin{definition}
  A metrics function takes a ranking and a known object and yields a rational number
  that represents how the ranking relates to the actual ranking for the known object.
  Such a function is of the type $\rankings \rightarrow \objs \rightarrow \rat$.
\end{definition}

\begin{definition} The set of dependencies required to solve conjecture $\type[s]$ is defined as
  \[ \required[s] = \deps{s} \]
\end{definition}

\begin{definition} The set of suggestions done in a ranking is defined as
  \[ \suggestions{r} = \{ d \in \depset ~|~ r(d) ~\text{is defined} \} \]
\end{definition}

\begin{definition} We define $\topn{n}{r}$ to be
  \[ \topn{n}{r} = \downset{\depset}{\nth{n}{(\suggestions{r}, \lambda x~y. r(x) > r(y))}} \]
\end{definition}

\begin{definition}
  Given a totally ordered set $(X, \leq)$, and an element $x \in X$, we define $\findindex{X}{x}$ to be the position of $x$ in totally ordered set $X$.
  \[ \findindex{X}{x} = \left\{
    \begin{array}{ll}
      0 & \text{if~} x = \infimum_X \\
      \findindex{X \cap \infimum_X}{x}+1 & \text{if~} x \neq \infimum_X \\
    \end{array}
    \right.
  \]
\end{definition}

In this thesis we use the following metrics:
\begin{definition}[\oocover]
  The recall of the set of proof dependencies by the first 100 suggestions is defined as
  \[ \oocoverf{r}{s} = \frac{ |\topn{100}{r} \bigcap \required[s]| } { |\required[s]| } \]
  This is our primary metric for evaluating the performance of our premise selection algorithms.
  An higher recall is better.
\end{definition}

\begin{definition}[\ooprecision]
  The precision of the set of proof dependencies by the first 100 suggestions is defined as
  \[ \ooprecisionf{r}{s} = \frac{ |\topn{100}{r} \bigcap \required[s]| } { |\topn{100}{r}| } \]
  An higher precision is better.
  Typically by making less suggestions a better precision is achieved.
  For the corpii we evaluated most of our algorithms yield more than 100 suggestions per proof goal.
  Thus this metric should often yield around the same value for differing algorithms.
\end{definition}

\begin{definition}[\recall]
  The \recall is defined to be the number of guesses that were made until the entire set of required dependencies is suggested.
  When the set of guesses does not include all required dependencies $\recallf{r}{s}$ yields $|\suggestions{r}|+1$,
  or the number of suggestions plus one.
  The function `$\firstsym$' is defined in Definition \ref{def:first}.
  \[ \recallf{r}{s} = \firstsym(\lambda i. (i \leq |\suggestions{r}|) \rightarrow \required[s] \subseteq \topn{i}{r}) \]
  A lower \recall is better.
\end{definition}

\begin{definition}[\rank]
  The \rank is defined to be the average position of an actual proof dependency in the sequence of suggestions ordered by predicted relevance.
  \[ \rankf{r}{s} = \frac{
      \sum\limits_{r \in \required[s]} \left\{
        \begin{array}{ll}
          \findindex{\suggestions{s}}{r} & \text{if~} r \in \suggestions{s} \\
          0 & \text{otherwise} \\
        \end{array}
        \right.
    }{
      |\suggestions{r} \bigcap \required[s]|
    }
  \]
  A lower \rank is better.
\end{definition}

\begin{definition}[Area Under Curve (AUC)]
  The area under the ROC Curve is defined to be the probability that a correctly suggested premise is ranked \emph{better} than an incorrectly suggested premise.
  Given $X$ to be the correctly suggested premises $X = \suggestions{r} \bigcap \required[s]$ and $Y$ to be the incorrectly suggested premises $Y = \suggestions{r} - \required[s]$ we define the \auc to be:
  \[
    \aucf{r}{s} = \frac{
      \sum_x^X \sum_y^Y {\findindex{\suggestions{s}}{x} < \findindex{\suggestions{s}}{y}}
    }{
      |X| \times |Y|
    }
  \]
  An higher AUC indicates a better performance.
\end{definition}

\begin{definition}[\volume]
  The volume is defined to be the number of suggestions made by a predictor.
  From a runtime performance standpoint it is better to make less suggestions, with the risk of missing relevant facts.
  When just the first 100 suggestions are looked at, this metric is of limited importance.
  \[
    \volumef{r}{s} = |\suggestions{s}|
  \]
\end{definition}

When reporting these metrics for the entire testset \testset~the average is taken for every computed prediction.
Technically we are then refering to the Mean \recall and Mean \rank, but for simplicity sake we will refer to
just '\recall' in the following sections when talking about the results for entire testsets.

\subsubsection{Varying terminology}
This thesis uses concepts from both the \ltr academic field (such as with \adarank)
as the premise selection / ATP with machine learning fields (as with the work by Kaliszyk).
However between these fields the terminology used is not always consistent.
This is especially the case for the metrics used, as some cross polination has occurred between the fields,
but where differing concepts have ended up with the same name.
To clarify this I've tabularized the terms used:

\begin{figure}[H]
  \centering
  \begin{tabular}{l|l}
  \ltr          & \emph{ATP with Machine Learning} \\\hline
  \oocover      & 100Cover \\
  \ooprecision  & 100Precision \\
  \recall       & Recall \\
  \rank         & Rank \\
  \auc          & \auc \\
  \end{tabular}
  \caption{The varying terminology used between academic fields.}
\end{figure}