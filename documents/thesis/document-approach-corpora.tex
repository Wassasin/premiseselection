A corpus can have different characteristics, which might reveal strengths or weaknesses of predictors.
The following corpora are used as test cases:

\begin{figure}[H]
  \begin{tabular}{lllrrr}
    Corpus & commit & release date & $|S|$ & $|\defs|$ & $|\thms|$ \\\hline
    \coq & b705cf0 & april 2015 & 13687 & 4080 & 13037\\
\formalin & 64d98fa & november 2015 & 19153 & 5366 & 18071\\
\mathclasses & 751e63b & june 2016 & 16080 & 5125 & 15268\\
\corn & 4860de7 & september 2009 & 24033 & 7058 & 21559\\
\mathcomp & 9e81c8f & december 2015 & 31109 & 7745 & 27641\\

  \end{tabular}
  \caption{Corpora examined in this thesis, along with the relevant version, release date and number of objects, definitions and theorems.}
\end{figure}

All code was compiled using \coq version 8.4pl6.
Newer versions no longer support the XML export plugin, and as such do not support the workflow of \roerei as of yet.

\begin{description}
  \item[\coq stdlib]
  \item[\formalin]
    (or $ch_2o$)
  \item[\mathclasses] \cite{spitters2011type} \cite{krebbers2011type} \cite{krebbers2011computer}
    A library of abstract interfaces for mathematical structures in \coq.
    It is heavily based on Coq's Type Classes \cite{sozeau2008first} in order to share notations by overloading and specifying with abstract structures by quantification on contexts.
    It provides interfaces for algebraic hierarchies (groups, rings, field, \ldots),
    relations, orders, categories, functors, universal algebra,
    numbers ($\mathbb{N}$, $\mathbb{Z}$, $\mathbb{Q}$, \ldots)
    and operations (shift, power, abs, \ldots).
    With it, \mathclasses provides structure inference, multiple inheritance, convenient algebraic manipulation
    and idiomatic use of notations.
    It contains relatively the least amount of theorems compared to the other corpora.
  \item[\corn]
    The \coq Constructive Repository at Nijmegen is a corpus comprised of various different projects developed (partially) at the Radboud University Nijmegen.
    It depends on \mathclasses for the implementation of high performance exact real number arithmetic.
    A relatively older version of \corn is examined, the same as examined by Kaliszyk \cite{kaliszyk2014machine}.
    This in order to attempt to provide a fair comparison of our experimental results.
    Currently it is (roughly) comprised of:
    \begin{itemize}
      \item Fundamental Theorem of Algebra and the algebraic hierarchy. \cite{geuvers2002constructive}
      \item Fundamental Theorem of Calculus. \cite{cruz2002constructive}
      \item Program extraction for real computation. \cite{cruz2006large}
      \item Abstract model of the real numbers. \cite{niqui2008coinductive}
      \item Efficient computation with real numbers and metric spaces. \cite{o2008certified}
      \item Riemann integration. \cite{o2010computer}
      \item Interface with Coq's standard library reals. \cite{kaliszyk2008computing}
      \item The ForMath project, for which the \mathclasses library was developed. \cite{spitters2011type}
        It includes an Ordinary Differential Equations solver. \cite{makarov2013picard}
    \end{itemize}

  \item[\mathcomp]
\end{description}

\begin{figure}[H]
	\centering
	\begin{tikzpicture}[auto, node distance=2.5cm, main/.style={align=center}]
		\node[main] (coq) {\coq};
    \node[main] (mathclasses) [right=2cm of coq] {\mathclasses};
    \node[main] (formalin) [above=0.5cm of mathclasses] {\formalin};
		\node[main] (mathcomp) [below=0.5cm of mathclasses] {\mathcomp};
    \node[main] (corn) [right=1.5cm of mathclasses] {\corn};

    \draw[->] (coq) to [out=0,in=180] (formalin);
    \draw[->] (coq) to [out=0,in=180] (mathcomp);
    
    \draw[->] (coq) to [out=0,in=180] (mathclasses);
    \draw[->] (mathclasses) to [out=0,in=180] (corn);
    
	\end{tikzpicture}
	\caption{Dependencies between corpora.}
	\label{figure:corporadeps}
\end{figure}

\todo{Are dependencies loaded?}