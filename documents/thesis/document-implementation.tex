\label{section:implementation}

In this section we will discuss considerations we made during the implementation of \roerei.
These considerations are not very interesting from an academic standpoint, but moreso from an engineering standpoint.
Also students might be interested.

We started the \roerei implementation in OCaml.
This seemed to be a great choice for a programming language as it is easy to integrate with CoqIDE in later stage.
However problems arose during this early implementation:
\begin{itemize}
\item (De)serialization of the various datasets takes an unacceptable amount of time. (3 minutes vs 1 second in C++).
\item The standard library implementation for List is not tail recursive, resulting in stack overflows for large objects.
    It is a wonder that anyone uses this implementation.
\item Any 3rd party libraries using the standard library thus are unsuitable.
\end{itemize}

Thus we quickly decided to continue the implementation in C++.
Currently only the first parser and resolver stages of \roerei are implemented in OCaml.
The parser shares the \acic type definitions from the \xml plugin in \coq 8.4.


Quite a bit faster for everything.
Currently KNN CV takes 4.5 minutes
