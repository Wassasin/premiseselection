During compilation of a \coq project the \xml plugin has the capability to emit \xml files containing \acic objects.
This plugin is shipped with \coq by default until version 8.4pl5.
For later versions there is currently no equivalent functionality available for \coq.
To instruct the plugin to emit \xml the following variables need to be set when calling \texttt{coqc} (for example via a Makefile):
\begin{center}
	\lstinline{COQ_XML=-xml COQ_XML_LIBRARY_ROOT=<dest>}
\end{center}

For some corpora it can be troublesome to have the compiler emit proper \xml.
We have developed several patches for the datasets used in this thesis.
For more information on the specific versions and patches used,
view Section \ref{section:corpora} and the \texttt{data} folder in the repository for this thesis.

From the parsed \xml files only the following \acic constructs are used:
\begin{description}
    \item[(Co)Inductive definitions]
        These definitions are composed of a name, a type, and a list of constructors.
        Each constructor also is composed of a name and a type.
		A (Co)Inductive definition also yields an induction principle (\texttt{ind}) and a recursion scheme (\texttt{rec}) as separate terms.
    \item[Constants (definitions / theorems / axioms)]
		The types and bodies (the body is equivalent to the \acic term) of constants are defined in separate \xml files.
		This separation is an optimization, as once defined the proofs (terms) of theorems are irrelevant.
		For \coq it is sufficient to only know the type of a theorem.
		For definitions the term is absolutely necessary, as the definitions needs to be substituted when applied.
		However, conceptually theorems and definitions become indistinguishable when exported (see Section \ref{section:thmsdefs}).
        Axioms only generate a type, as the term of an axiom is empty.
\end{description}

The following objects are not used, but could be useful in future work:
\begin{description}
    \item[Proof in progress]
		Consists of a name, a type, a body and a list of dependencies (also called subgoals).
        These dependencies still need to be satisfied to complete the proof.
    \item[Tactics application]
        On a higher level, tactics can be applied in order to form a proof.
        These higher level constructs are dependent on the proof engine.
        A proof consisting of tactics can thus become invalid given another \coq version.
        For premise selection these tactics can help solve proofs more quickly, because the proofs they form are smaller.
\end{description}

\coq also exports variables, but these are not used.
A variable consists of a name and a type, and becomes a parameter when a theory which uses the variable is applied.
Thusfar there is no use for these variables within this thesis.

\subsubsection{Formal extraction}
In the repository \xml directory we have for each \coqobj[s] comprised of $\name \objdef \term : \type$
one or two files: at least a file ending with \lstinline{.types.xml.gz} and
possibly a file ending with \lstinline{.body.xml.gz} if the body is not empty, i.e. $\emptyterm$.
The definition of the constructors as used by the parser and resolver can be found
in the \roerei repository in the file \lstinline{src/preloader/acic.ml}.
The original source code for these functions can be found in \lstinline{src/preloader/interpreter.ml}.
These functions are paraphrased here using some bare math.
As these functions are programmed in \ocaml, we use a functional style to describe them.
Before we can continue, we first need to define a list type $\listtype{-}$ and some operations on this type.

\begin{definition}[\mapsymbol]
  The $\mapsymbol$ function transforms a list of type $\listtype{X}$ to a list of type $\listtype{Y}$ using a function $f$.
  \[ \mapsymbol : \forall X, Y ~.~ \listtype{X} \rightarrow (X \rightarrow Y) \rightarrow \listtype{Y} \]
  \[ \begin{array}{ll}
      \map{\nil}{f} & = \nil \\
      \map{\cons{x}{l}}{f} & = \cons{f(x)}{\map{l}{f}} \\
    \end{array} \]
\end{definition}

\begin{definition}[\concatsymbol]
  Two lists can be concatenated using the infix $\concatsymbol$-function.
  \[ \concatsymbol : \listtype{X} \rightarrow \listtype{X} \rightarrow \listtype{X} \]
  \[ \begin{array}{ll}
      \concat{\nil}{y} & = y \\
      \concat{\cons{a}{x'}}{y} & = \cons{a}{\concat{x'}{y}} \\
    \end{array} \]
\end{definition}

\begin{definition}[\singleton{-}]
  We use a shorthand to denote lists with only one element:
  \[ \singleton{x} = \cons{x}{\nil} \]
\end{definition}

\begin{definition}[\flattenlistsym]
  Also a flatten operation can be defined to convert lists of lists $\listtype{\listtype{X}}$ to singular lists $\listtype{X}$.
  \[ \flattenlistsym : \forall X . \listtype{\listtype{X}} \rightarrow \listtype{X} \]
  \[ \begin{array}{ll}
      \flattenlist{\nil} & = \nil \\
      \flattenlist{\cons{x}{l}} & = \concat{x}{\flattenlist{l}} \\
    \end{array} \]
\end{definition}

\begin{definition}[\foldrsym]
  Finally we also use the folding operation to iterate over lists and gather a result using a callback function.
  \[ \foldrsym : \forall X Y ~.~ (X \rightarrow Y \rightarrow Y) \rightarrow Y \rightarrow \listtype{A} \rightarrow Y \]
  \[ \begin{array}{ll}
      \foldr{f}{z}{\nil} & = z \\
      \foldr{f}{z}{\cons{x}{l}} & = f(x,~ \foldr{f}{z}{l})
    \end{array}\]
\end{definition}

For each \coqobj[s] we count the names occuring in either the type or the term of the object.
With each name we also yield the depth the name occurs.
This count function $c$ is rather straightforward.
A specific flatten operation $\flattensym$ can transform this list of occurances to the desired summary $<\name[s], \termset{s}, \typeset{s}>$.

\begin{definition}[$c$]
  We define a count function $c$ on \acic constructor terms yielding tuples of names and their depth in the \acic expression:
  \[c : \terms \rightarrow \nat \rightarrow \listtype{\names \times \nat} \]

  \begin{tabularx}{\textwidth}{lll}
    $c(\emptyterm, d)$              & $ = \nil $ & Empty term \\
    $c(\text{ARel}(\cdots), d)$     & $ = \nil $ & \todo{unclear} Unused \\
    $c(\text{AVar}(u), d)$          & $ = \singleton{(u, d)} $ & (Local) variables \\
    $c(\text{AEvar}(l), d)$         & $ = \flattenlist{\map{l}{\lambda x . c(x, d+1)}} $ & \todo{unclear} \\
    $c(\text{ASort}(\cdots), d)$    & $ = \nil $ & The Sort of types, unused \\
    $c(\text{ACast}(t, A), d)$      & $ = \concat{c(t, d+1)}{c(A, d+1)} $ & Type casting check \\
    $c(\text{AProds}(l, t), d)$     & $ = c_{\times}(l, t, d) $ & $\prod l . t$ \\
    $c(\text{ALambdas}(l, t), d)$   & $ = c_{\times}(l, t, d) $ & $\lambda l . t$ \\
    $c(\text{ALetIns}(l, t), d)$    & $ = c_{\times}(l, t, d) $ & $\text{let~} n := a, \ldots \text{~in~} t$ \\
    $c(\text{AApp}(l), d)$  & $ = \left(
      \begin{array}{l}
        \singleton{(\text{special:app}, d)} ~\concatsymbol \\
        \map{l}{ \lambda a . c(a, d+1) }
      \end{array}
      \right)
      $ & Application $(x~y)~z$ \\
    $c(\text{AConst}(\cdots, u), d)$  & $ = \singleton{(u, d)} $ & Constant \\
    $c(\text{AInd}(\cdots, u), d)$  & $ = \singleton{(u, d)} $ & Inductive definition \\
    $c(\text{AConstruct}(\cdots, u), d)$  & $ = \singleton{(u, d)} $ & Constructor (of ind. def.) \\
    $c(\text{ACase}(u, A, i, l), d)$  &
      $= \left(
        \begin{array}{l}
          \concat{\singleton{(u, d)}}{
            \concat{c(A, d+1)}{c(i, d+1)}
          }
          ~\concatsymbol \\
          \flattenlist{ \map{l}{\lambda x . c(x, d+1)} } \\
        \end{array}
        \right)
      $
      & Deconstruction (of ind. types) \\
    $c(\text{AFix}(l), d)$        & $ = c_{\text{fix}}(l, d) $ & Fixpoint functions \\
    $c(\text{ACoFix}(l), d)$      & $ = c_{\text{fix}}(l, d) $ & Co-fixpoint functions \\
  \end{tabularx}
  \vspace{1em}\\
  with \[
    \begin{array}{l}
      c_{\times}(l, t, d) =
        \concat{c(t, d+1)}{
          \map{l}{ \lambda (n, a) ~.~ \cons{(\text{special:prod}, d)}{c(a, d+1)} }
        }\\
      c_{\text{fix}}(l, d) = \flattenlist{ \map{l}{ \lambda (A, t) ~.~ \concat{c(A, d+1)}{c(t, d+1)} } }
    \end{array}
    \]
\end{definition}

\begin{definition}[$c_{\text{type}}, c_{\text{term}}$]
  We define the counting of types and terms for top level \coqobjs to be:

  \vspace{1em}
  \begin{tabular}{ll}
    $c_{\text{type}}(\text{AConstant}(n, A, t))$ & $= c(A, 0)$ \\
    $c_{\text{type}}(\text{AInductiveDefinition}(n, l))$ & $= \flattenlist{\map{l}{ \lambda (A, m) ~.~ \concat{c(A, 0)}{\flattenlist{\text{map}(m, \lambda x . c(x, 0))}}} } $\\
    $c_{\text{type}}(-)$ & $= \nil $ \\
    & \\
    $c_{\text{term}}(\text{AConstant}(n, A, t))$ & $= c(t, 0)$ \\
    $c_{\text{term}}(-)$ & $= \nil $ \\
  \end{tabular}
\end{definition}

Finally we can define the various flatten operations on terms:
\begin{definition}[$\flatten{x}$]\defgls{flatten}
  The occurances of names can be extracted with the following:
  \[ \flattensym : \terms \rightarrow 2^{\names} \]
  \[ \flatten{t} = \foldr{\lambda (n, d)~s ~.~ \{n\} \cup s}{\varnothing}{c(t, 0)} \]
\end{definition}

\begin{definition}[$\countoccur{x}$]\defgls{countoccur}
  The number of times a name $n$ occurs can be counted as follows:
  \[ \countoccursym : \terms \rightarrow \nat^{\names} \]
  \[ \countoccur{t} = \lambda n ~.~ \foldr{\lambda (m, d)~s ~.~ \text{if~} n = m \text{~then~} s+1 \text{~else~} s}{0}{c(t, 0)} \]
\end{definition}

\begin{definition}[$\depthoccur{x}$]\defgls{depthoccur}
  The minimum depth a name $n$ occurs can be determined with function $\depthoccursym$.
  Contrary to the counting function $\countoccursym$ there is no natural way to denote that a name does
  not occur in a term.
  That is why $\depthoccursym$ is a partial function.
  
  \[ \depthoccursym : \terms \rightarrow \names \rightharpoonup \nat \]
  \[
    \begin{array}{l}
      \depthoccur{t} = \lambda n ~.~ \foldr{\lambda (m, d)~s ~.~ \text{if~} n = m \text{~then~} f(d, s) \text{~else~} s}{\infimum}{c(t, 0)} \\
      \\
      \text{with~} \begin{array}{l}
        f(d, \infimum) = d \\
        f(d, s) = \text{min}(d, s) \\
      \end{array}
    \end{array}
  \]
\end{definition}